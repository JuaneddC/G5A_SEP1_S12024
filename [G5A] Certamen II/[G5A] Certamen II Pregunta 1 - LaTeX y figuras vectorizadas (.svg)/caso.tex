\documentclass[conference]{IEEEtran}
\IEEEoverridecommandlockouts
\usepackage{cite}
\usepackage{amsmath,amssymb,amsfonts}
\usepackage{algorithmic}
\usepackage{amsmath}
\usepackage{graphicx}
\usepackage{textcomp}
\usepackage{xcolor}
\def\BibTeX{{\rm B\kern-.05em{\sc i\kern-.025em b}\kern-.08em
    T\kern-.1667em\lower.7ex\hbox{E}\kern-.125emX}}
\begin{document}

\title{Pregunta 1 Certamen N°2\\
{\footnotesize \textsuperscript{*}Análisis de Sistemas Eléctricos de Potencia}
\thanks{Universidad Técnica Santa María}
}

\author{\IEEEauthorblockN{1\textsuperscript{st} Juan Pérez Rojas}
\IEEEauthorblockA{\textit{Departamento de Ingeniería Eléctrica} \\
Valapraíso, Chile \\
juan.perezr@usm.cl}
\and
\IEEEauthorblockN{2\textsuperscript{nd} Nicolás Garrido Boggioni}
\IEEEauthorblockA{\textit{Departamento de Ingeniería eléctrica}\\
Valparaíso, Chile \\
nicolas.garridob@usm.cl}}

\maketitle


\section{Introducción}
En este informe se resuelven problemas relacionados con la alimentación de un tren eléctrico mediante la obtención y evaluación
de parámetros de la línea eléctrica que lo abastece.
\begin{center}
\includegraphics[width=0.4\textwidth]{tren 4.0.png}\\
\small{Fig. 1: Tren a analizar.}
\end{center}


\section{Formulación matemática del problema}
Este problema se puede modelar por la siguente ecuación:
$$
\begin{pmatrix}
V_{s}\\ 
I_{s}
\end{pmatrix}
= 
\begin{pmatrix}
1 & 0\\ 
Y_{c}& 1
\end{pmatrix}
\begin{pmatrix}
A & B\\ 
C & D
\end{pmatrix}
\begin{pmatrix}
1 & 0\\ 
Y_{c}& 1
\end{pmatrix}
\begin{pmatrix}
V_{r}\\ 
I_{r}
\end{pmatrix}
$$

\begin{center}
\small {Ec. 1: Modelamiento matricial de línea con compensación shunt.}
\end{center}


\indent Esta relaciona los valores de entrada a una línea y los valores de salida dependiendo de los parámetros de la línea, e implícitamente
también a los parámetros de la carga a través de la potencia de salida.
\section{Variación de la magnitud de tensión}
La tensión varía a lo largo de la línea debido a su disposición y al material usado. En este caso se usará el modelo $\pi$ para el desarrollo
del problema sin compensación, se define:
$$
P_{r}=0.25\left [\text{\small $\frac{kWh}{km\cdot pasaj}$}\right ]\cdot100\left [\text{\small $\frac{km}{h}$}\right ]\cdot200[pasaj]=5[MW]
$$
\begin{center}
\small{Ec. 2: Cálculo de potencia.}
\end{center}
Ahora se puede usar una ecuación desprendida de las matrices:
$$
V_{s}=A\cdot V_{r}+B\cdot I_{r}
$$
\begin{center}
\small{Ec. 3: Forma general del voltaje de entrada.}
\end{center}
Luego, reemplazando:

$$
V_{s}^{2}=\left( \mathbb{Re}\left \{ A\cdot V_{r} \right \} \right )^{2}+\left(\mathbb{Re}\left \{ B\cdot \frac{P_{r}}{V_{r}} \right \}\right )^{2}+\left(\mathbb{I}\left \{ A\cdot V_{r} \right \}\right )^{2}+...
$$$$
...+\left(\mathbb{I}\left \{ B\cdot \frac{P_{r}}{V_{r}} \right \}\right )^{2}+...
$$$$
 ...+2\left( \mathbb{Re}\left \{ A\cdot V_{r} \right \} \cdot\mathbb{Re}\left \{ B\cdot \frac{P_{r}}{V_{r}} \right \}+\mathbb{I}\left \{ A\cdot V_{r} \right \}\cdot\mathbb{I}\left \{ B\cdot \frac{P_{r}}{V_{r}} \right \}\right)
$$
\begin{center}
\small{Ec. 4: Forma expandida de Ec. 3.}
\end{center}
Para finalemente encontrar que:
$$
V_{r}=17.17[kV] \rightarrow  \Delta \left | V_{s,r} \right| =5.33[kV]
$$

\begin{center}
\small{Ec. 5: Resultados.}
\end{center}
\section{Estabilidad teórica}
Es posible ver que, cuando el límite de estabilidad teórico es máximo, su ángulo tendrá un valor de alredor 90°, en cuanto al mínimo es
posible considerar un angulo entre 0° y 1°.
$$
P_{max,min} = \frac{V_{s} \cdot V_{r}}{{X}'}\cdot sin(\delta )
$$
\begin{center}
\small{Ec. 6: Ecuación de potencia para estabilidad teórica.}
\end{center}
$$
\left. P_{max} = \frac{22.5 [kV] \cdot 17.17 [kV]}{34.75 [\Omega]}\cdot sin(90^{\circ}) =
10.806[MW]
\atop P_{min} = \frac{22.5 [kV] \cdot 17.17 [kV]}{34.75 [\Omega]}\cdot sin(1^{\circ}) = 188.595[kW]\right\}
$$
\begin{center}
\small{Ec. 7: Resultados reemplazando datos en Ec.6.}
\end{center}

\section{Compensación shunt}
\subsection{Método alternativo}
El principal inconveniente es el espacio y peso adicional. Una alternativa son los bancos de condensadores en las subestaciones eléctricas o en las torres de transmisión del tren.
\subsection{Compensación dinámica}
La compensación para elevar el voltaje de salida puede hacerse mediante capacitores en paralelo a la fuente y la carga, los que en realidad serían los convertidores de potencia, quienes variarían estos valores y por ende la potencia reactiva dependiendo de la distancia del tren a la fuente.
\begin{center}
\includegraphics[width=0.4\textwidth]{csix.png}\\
\small {Fig. 2: Sistema simplificado de una línea (rosa) y convertidores(celeste).}
\end{center}


\end{document}
